\documentclass[oneside,12pt]{memoir}
\usepackage{makeidx}
%\usepackage[utf8]{inputenc}
%\pagestyle{plain}


%%%%%%%%%%%%%%%%%%%%%%%%%%%%%%%%%%% importa pacchetti
\usepackage{usepkg}
%%%%%%%%%%%%%%%%%%%%%%%%%%%%%%%%%%%%%

\addtocontents{toc}{\cftpagenumbersoff{part}} %% part senza numero pagina

%%%%%%%%%%%%%%%%%% titletoc, titlesec setting.
%%%%%%%%%%%%%%%%%%      pagestyle
\usepackage{titleT}
\pagestyle{plain}
%%%%%%%%%%%%%%%%%% lenth
\usepackage{length}

%%%%%%%%%%%% Hyperref package
\usepackage{hyperref}
\hypersetup{
    colorlinks,
    citecolor=black,
    filecolor=black,
    linkcolor=black,
    urlcolor=black
}
%%%%%%%%%%%%%%%%%%%%%%%%

%%%%%%%%%%%%%%%%%Geometry package
\usepackage[a4paper,margin=40px,tmargin=20px,includefoot,footskip=40px]{geometry}

%http://tex.stackexchange.com/questions/211248/problem-with-cropmarks-on-geometry-package
%http://www.ctex.org/documents/packages/layout/geometry.pdf

%%%%%%%%%%%%%%%%%%%%%%%%%%%%%%%%%%% Funzioni generali
\usepackage{functions}
%http://tex.stackexchange.com/questions/246/when-should-i-use-input-vs-include
\usepackage{sources}
%%%%%%%%%%%%%%%%%%%%%%%%%%%%%%%%%%% Funzioni per questo file main
\usepackage{mathOp}
\usepackage{LocalF}
%%%%%%%%%%%%%%%%%%%%%%%%%%%%%%%%%

\makeindex
\raggedbottom %http://tex.stackexchange.com/questions/102084/annoying-paragraph-spacing-issue-with-memoir

\author{ }
\title{Studio delle oscillazioni solari}
\date{}

\begin{document}

\maketitle


\tableofcontents



\frontmatter

\begin{abstract}


Il sole \'e una massa di gas autogravitante che supporta numerosissimi modi di oscillazione attorno alla sua posizione di equilibrio. Il loro studio fornisce uno strumento per determinare caratteristiche essenziali della struttura solare. In seguito alla comprensione della struttura spettrale delle oscillazioni solari sono state identificate altre stelle che mostrano oscillazioni con analoga struttura.

\begin{comment}
Analizzer\'o inizialmente le grandezze astrofisiche fondamentali, le importanti relazioni tra di esse, e le grandezze fondamentali che evidenziano i fenomeni fisici dominanti; passer\'o poi alla descrizione dei dati osservativi riguardo alle oscillazioni solari.
\end{comment}

La rivelazione dei moti periodici della fotosfera solare (oscillazione dei 5 minuti: \cite{lei62velocity}) e la scoperta, in misure in cui la superficie solare \'e risolta spazialmente (\cite{deu75observations}) e in misure integrate sull'intero disco solare (\cite{cla79solar}), che tali moti periodici sono la sovrapposizione di modi discreti, sono le basi osservative dell'eliosismologia. 
In questo elaborato discuto molto brevemente le osservazioni relative  alle oscillazioni con periodo 5 minuti e la loro struttura modale e le osservazioni di Duvall (\cite{duv82dispersion})(, che mostrano la relazione tra le propriet\'a superficiali delle oscillazioni e il loro comportamento all'interno del sole).

Accenner\'o brevemente alle tecniche osservative ed alle problematiche legate alla precisione richiesta dalle osservazioni eliosismologiche.

Lo scopo principale di questa tesina \'e invece una descrizione teorica dei modi di oscillazione del sole, risolvendo con tecniche approssimate le equazioni che governano l'evoluzione di perturbazioni infinitesime dello stato di equilibrio; quindi quali propriet\'a della struttura solare li determinino direttamente e come si possano confrontare, attraverso alcune tecniche di inversione di base, i parametri caratteristici di un modello solare con i dati eliosismologici.

Introduco quindi i modi normali per il moto ondoso: suppongo che le grandezze fisiche che determinano il problema dipendano solo dalla distanza dal centro, \'e quindi naturale descrivere l'ampiezza delle oscillazioni  in termini di armoniche sferiche per la dipendenza angolare che sono identificate dalla distribuzione caratteristica delle fasi di oscillazione sulla superficie solare. Queste definiscono il grado l del modo normale; le autofunzioni dell'ampiezza dell'oscillazione radiale sono caratterizzate dall'indice n, il cui modulo riflette il numero di zeri dell'ampiezza radiale.

La piccola ampiezza delle oscillazioni mi permette di usare la teoria delle perturbazioni lineari applicata alle equazioni di un un corpo autogravitante in equilibrio idrostatico per ricavare attraverso l'equazione del moto (equazione di Eulero) un'equazione vettoriale agli autovalori per le frequenze di pulsazione. Questo tipo di problema \'e comune in fisica teorica quindi esistono molte tecniche numeriche per determinare le frequenze con l'accuratezza necessaria per il confronto con i dati sperimentali ma non tratter\'o questa problematica.


Ricavo il sistema di equazioni differenziali che descrive le perturbazioni adiabatiche e scrivo la relazione di dispersione nella forma pi\'u generale: distinguo una zona in cui si propagano i modi acustici o modi p, con frequenza maggiore della frequenza di Lamb, ed una zona, la zona interna, in cui si propagano i modi g o modi di gravit\'a aventi frequenza minore della frequenza di Brunt-\vai{}, e modi f confinati in superficie. 

Uso le leggi dell'acustica geometrica per stimare lo spessore in cui sono confinati i modi acustici: dalla relazione di dispersione per onde acustiche deduco la distanza dal centro del sole tale che il moto ondoso sia puramente orizzontale, in quel punto ho che la frequenza delle oscillazioni \'e uguale alla frequenza di Lamb. I modi normali di oscillazione della superficie sono prodotti dall'interferenza di un gran numero di onde aventi in comune la distanza dal centro di inversione del moto. Da calcoli accurati risulta che i modi p sono confinati nella parte esterna della zona convettiva. 

I risultati eliosismologici dimostrano la validit\'a dei modelli solari standard. Di particolare importanza \'e la  misura  dell'abbondanza di elio nel sole, valore che nei modelli solari standard viene variato, insieme al parametro che regola l'efficienza del trasporto energetico nella zona convettiva, per ottenere, determinando numericamente l'evoluzione del modello iniziale, i giusti valori di luminosit\'a e raggio attuali.

Infine mostro come l'inversione eliosismologica fornisca una guida per individuare le zone in cui il modello solare appare non completamente corretto.

\end{abstract}


\mainmatter

\part{Fondamenti dell'eliosismologia.}

\chapter{Introduzione alle oscillazioni solari.}

L'osservazione di fenomeni periodici nelle stelle permette di dedurre informazione sulla loro struttura.

Le stelle in sequenza principale, per cui il contributo dell'energia nucleare bilancia il flusso di energia verso l'esterno e la condizione di equilibrio idrostatico \'e verificata in ogni strato della stella, sono caratterizzate da numerosi modi di oscillazione di piccola ampiezza cio\'e onde gravo-acustiche stazionarie le cui caratteristiche sono determinate dalle caratteristiche della stella.

Lo studio delle oscillazioni della superficie solare e l'estrapolazione delle informazioni sulla struttura interna in essi contenuta \'e detta eliosismologia: le frequenze sono determinate principalmente dalla stratificazione (e dinamica) della regione in cui le ampiezze sono apprezzabili.

\'E possibile calcolare numericamente le frequenze sulla base di un modello stellare e al variare di uno o pi\'u parametri del modello analizzare la corrispondenza con quelle osservate inoltre sono state sviluppate tecniche di inversione per valutare le discrepanze tra modello solare e realt\'a dalla differenza tra frequenze teoriche e osservate.


\section{Equilibrio idrostatico.}

La costanza delle caratteristiche solari su tempi dell'ordine del \si{\mega\year} conferma l'ipotesi che il Sole sia in equilibrio .

\subsection{Distribuzione della massa.}

La distribuzione di massa del Sole \'e determinata dall'equilibrio tra la forza di attrazione gravitazionale e il gradiente della pressione del gas: considero una distribuzione di massa sferica, la massa presente in un guscio infinitesimo \'e descritta da

\begin{align*}
&dm=4\pi r^2\rho \,dr-4\pi r^2\rho v\,dt&\intertext{equivalente all'equazione di continuit\'a}\\
&\PDy{t}{\rho}+\nabla\cdot(\rho\vec{v})=0&\intertext{e poich\'e consideriamo una configurazione di equilibrio statico, la velocit\'a radiale del guscio sferico $v=0$:}\\
&dm=4\pi r^2\rho \,dr
\end{align*}

\subsection{Condizione di equilibrio idrostatico}

La forza totale agente su un volume V di superficie S \'e
\begin{equation*}
\int_V\vec{f}\rho\,dV+\int_S\vec{t}(\vec{n},\vec{x},t)\,dS
\end{equation*}
dove $\hat{n}$ \'e la normale in ciascun punto di S, e $\vec{F}$ \'e una forza per uni\'a di massa, definisco il tensore degli sforzi $P\indices{_i_j}$ prendendo $\hat{n}=\hat{e}_i$
\begin{equation*}
\vec{P}_i=\vec{t}(\hat{e}_i,\vec{x},t)=(P_{i1},P_{i2},P_{i3})
\end{equation*}


La conservazione della quantit\'a di moto richiede
\begin{align}
&\rho\TDy{t}{v\indices{_i}}=-\partial\indices{_j}P\indices{_i_j}+\rho f\indices{_i}\nonumber&\intertext{dove $P_{ij}$ rappresentano le componenti del tensore degli sforzi: . Per pressione termodinamica cio\'e trascurando viscosit\'a molecolare e radiativa, campi magnetici e turbolenze}\nonumber\\
&\rho\TDy{t}{\vec{v}}=-\nabla P+\rho\vec{f}\nonumber&\intertext{ottengo quindi la condizione di equilibrio idrostatico $\ddvec{r}=0$:
}\nonumber\\
&\nabla P=\rho \vec{f}\label{eq:idrosta}
\end{align}


\subsection{Potenziale gravitazionale.}

Esplicito la forma della forza per unit\'a di massa f:
\begin{equation}\label{eq:gravitya}
g=\frac{Gm(r)}{r^2}
\end{equation}
diretta verso il centro di massa.

Il potenziale gravitazionale \'e soluzione dell'equazione di Poisson 
\begin{align}
&\nabla^2\Phi=4\pi G\rho\label{eq:poisson}\\
&\vec{g}=-\PDy{r}{\Phi}=\frac{Gm(r)}{r^2}\hat{r}
\end{align}

Sostituendo nell'equazion di equilibrio idrostatico
\begin{equation}
\TDy{r}{P}=-\frac{Gm(r)\rho(r)}{r^2}\label{eq:idrostae}
\end{equation}


\subsection{Tempo di evoluzione dinamico.}

Scrivo l'equazione del moto per la superficie unitaria di un guscio sferico
\begin{align}
&\frac{dm}{4\pi r^2}\PtwoDy{t}{r}=f_P+f_g\nonumber\\
&\frac{1}{4\pi r^2}\PtwoDy{t}{r}=-\PDy{m}{P}-\frac{Gm}{4\pi r^4}\label{eq:motionshell}
\end{align}

Il valore tipico della derivata \'e approssimato dal rapporto delle quantit\'a quindi
\begin{align*}
&\tau_{ff}\approx\sqrt{\frac{R}{g}}&\intertext{tempo caratteristico di una distribuzione sferica di materia in caduta libera cio\'e considerando solo il secondo termine in \eqref{eq:motionshell},}\\
&\tau_{esp}\approx R\sqrt{\frac{\rho}{P}}&\intertext{tempo caratteristico di espansione dovuta al termine di pressione esclusivamente.}
\end{align*}

Nelle stelle in cui l'equilibrio idrostatico \'e una buona approssimazione il tempo caratteristico di reazione a perturbazione dell'equilibrio idrostatico \'e

\begin{align*}
&\tau_{idro}\approx \sqrt{\frac{R^3}{GM}}\approx\frac{1}{2}(G\overline{\rho})\expy{-\frac{1}{2}}\\
&G\msun=\num{1.32712440018e20}\pm\num{8e9}\si{\cubic\meter\per\square\second}\\
&\tau_{idro}^{\odot}\approx\SI{27}{\minute}
\end{align*}

\section{Teorema del viriale.}

Il teorema del viriale esprime una propriet\'a statistica di particelle interagenti: in particolare ricavo una relazione tra energia interna ed energia potenziale gravitazionale e quindi ricavo un valore caratteristico per la velocit\'a del suono all'interno del Sole. 

L'energia potenziale gravitazionale della stella \'e
\begin{equation}
\Omega=-\int_0^M\frac{Gm(r)}{r}\,dm\label{eq:energiapg}
\end{equation}

e il teorema del viriale implica che

\begin{align*}
&\frac{1}{2}\TtwoDy{t}{I}=2K+\Omega&\intertext{con K energia cinetica totale:}\\
&K=\frac{1}{2}\sum_im_iv_i^2=\frac{1}{2}\sum_i\scap{p_i}{v_i}=\frac{3}{2}\int_VP\,dV=\frac{3}{2}\int_M\frac{P}{\rho}\,dm(r)
\end{align*}



Nel caso di un sistema in equilibrio idrostatico ($\frac{1}{2}\TtwoDy{t}{I}=0$) il teorema del viriale si riscrive
\begin{align*}
&0=\int_M\frac{3P}{\rho}\,dm(r)+\Omega
\end{align*}


Se il sistema \'e descritto dall'equazione di stato dei gas perfetti monoatomici ($\gamma=\frac{5}{3}$) ho

\begin{align}
&K=E_i=\int_0^Mu\,dm\nonumber&\intertext{e quindi}\nonumber\\
&\Omega=-2E_i\label{eq:virialegpm}
\end{align}

Per un'equazione di stato generale definisco il parametro $\zeta$
\begin{align*}
&\zeta u=3\frac{P}{\rho}&\intertext{per un gas ideale monoatomico (nel limite non relativistico)}\\
&\zeta=3(\gamma-1)\xrightarrow{\gamma=\frac{5}{3}}2
\end{align*}

se $\zeta$ \'e costante nella stella il teorema del viriale prende la forma

\begin{equation}
\zeta E_i+\Omega=0\label{eq:virialezetac}
\end{equation}


\section{Modo fondamentale di oscillazione.}

Le oscillazioni solari sono in prevalenza dovute ad onde acustiche, cio\'e oscillazioni in cui la forza di richiamo \'e prodotta dal gradiente di pressione, e quindi determinate dal profilo radiale della velocit\'a del suono.

I modi puramente acustici di un corpo finito o comunque con condizioni ai bordi sono onde stazionarie la cui parte reale \'e del tipo $f(x,y,z)\cos{(\omega t+\alpha)}$: in assenza di effetti dissipativi la velocit\'a di fase \'e nulla e la velocit\'a di gruppo infinita, velocit\'a e pressione sono sfasati di $\frac{\pi}{2}$.

Stimo il periodo del modo fondamentale di oscillazione solare.

Per un corpo in equilibrio idrostatico ricavo il valore medio della velocit\'a del suono utilizzando il teorema del viriale
\begin{align*}
    &-\Omega=3\int_VP\,dV=3\int_M\frac{P}{\rho}\,dm=3\int_M\frac{v_s^2}{\Gamma_1}\,dm\\
    &=3\exv{\frac{v_s^2}{\Gamma_1}}M\approx3\frac{\overline{v}_s^2}{\Gamma_1}M
\end{align*}

Se scrivo $\Omega=-q\frac{GM^2}{R}$, per stelle di sequenza principale ho che $q\approx1.5$:

per il modo fondamentale di oscillazione radiale
\begin{align*}
    &\lambda_1\approx 2\rsun{},\ \omega_1\approx\frac{c_s}{\lambda_1}\\
    &\Pi_1\approx\SI{1}{\hour}
\end{align*}


\section{Oscillazioni dei 5 minuti.}

In \citet{lei62velocity} si osserva che la superficie solare ha scale spazio-temporali privilegiate: in particolare \'e presente un comportamento periodico nell'atmosfera a tutte le altezze rilevato tramite effetto doppler. Il periodo \'e di circa 300 secondi e la lunghezza caratteristica di qualche \si{\mega\meter}.

Il modello proposto da \citet{ulrich70five} e \citet*{stein71five} considera le propriet\'a dei modi normali non radiali di oscillazione del Sole, in particolare dalla relazione di dispersione per onde acustiche si ha la definizione di cavit\'a risonanti al di sotto della superficie solare: le oscillazioni osservate hanno $\nu\geq\SI{500}{\micro\hertz}$, sono causate da modi p, onde stazionarie che si propagano come onda evanascente nell'atmosfera solare, e modi f di alto grado (onde di gravita di una superficie libera).

Sono possibili onde stazionarie per determinati valori di  $(k_h,\omega)$, dove $k_h=\sqrt{k_x^2+k_y^2}$ \'e il numero d'onda orizzontale avendo scritto il numero d'onda come $\vec{k}=k_r\hat{r}+\vec{k}_h$.

La simmetria sferica del modello solare rende naturale una descrizione delle perturbazioni in funzione di $Y\indices{_l^m}(\theta,\phi)$, l \'e l'ordine angolare: nell'approssimazione di onda piana si ha $k_h\approx\frac{l(l+1)}{r^2}$.


\subsection{Osservazione dei modi di alto grado angolare}

L'analisi tramite FFT (frequenza e numero d'onda) delle osservazioni della superficie solare riportate in \citet{deu75observations} confermano che la  potenza delle oscillazioni (con numero d'onda piccolo: $k_h=\frac{2\pi}{\lambda}<\SI{1}{\per\mega\meter}$) si distribuisce in linee determinate nel diagramma $(k_h,\omega)$ predette dal modello e quindi conferma che sono provocate da modi acustici non radiali degli strati interni alla fotosfera.

\subsection{Osservazione modi di basso grado angolare}

\citet{cla79solar} osservano nello spettro Doppler (Neutral K line: \SI{769.9}{\nano\meter}) della luce integrata sull'intero disco solare dei picchi equi-spaziati circa \SI{68}{\micro\hertz} interpretate come modi p di alto ordine n e basso grado l.

\subsection{Eccitazione e tempo di vita dei modi.}

Si ipotizza che le oscillazioni siano eccitate in maniera stocastica dai moti convettivi: la larghezza delle frequenze risonanti \'e determinata dal tempo di smorzamento.

\section{Cavit\'a risonanti.} % Spostare nella parte delle oscillazioni dopo condizioni al bordo per oscillazioni ??

Per la parte superiore dello spettro delle oscillazioni \'e una buona approssimazione la relazione di dispersione acustica 

\begin{equation*}
\omega^2=c^2(k_r^2+k_h^2)
\end{equation*}

Esistono delle regioni in cui \'e possibile moto oscillatorio per dati parametri dell'onda.
  
La superficie solare, a causa dell'aumento di $\omega_A=\frac{c}{2\densityscale{}}\sqrt{1-2\TDy{r}{\densityscale{}}}\propto T\expy{-\frac{1}{2}}$ che provoca la riflessione delle onde con periodo attorno ai 5-min, delimita superiormente la regione di propagazione.

L'aumento della velocit\'a del suono con la profondit\'a e la conseguente rifrazione dell'onda porta a propagazione del moto puramente tangenziale $k_r=0$: ci\'o avviene nel guscio sferico per cui $c_s=\frac{\omega}{k_h}\approx\omega \frac{r}{L}$ ovvero per $\omega=S_l$ frequenza di Lamb definita da $S_l^2=\frac{l(l+1)c^2}{r^2}$ \label{eq:Lambf}.

Maggiore \'e il grado l meno profonda \'e la cavit\'a: le cavit\'a acustiche si estendono nella zone convettiva fino alla profondit\'a in cui la rifrazione dovuta all'aumento della velocit\'a del suono $c\propto\sqrt{T}$ causa una riflessione totale dell'onda quando la velocit\'a del suono \'e aumentata fino alla loro velocit\'a di fase orizzontale.

Nella parte a basse frequenze dei modi g \'e valida una relazione di di spersione approssimata per $l\neq0$

\begin{align}
&k_r^2=\frac{S_l^2}{c^2}(\frac{N^2}{\omega^2}-1)\label{eq:dispersionag}&\intertext{ho introdotto la frequenza di \bv{}:}\nonumber\\
&N^2=g(\frac{1}{\Gamma_1P}\TDy{r}{p}-\frac{1}{\rho}\TDy{r}{\rho})=g(\frac{1}{\densityscale{}}-\frac{g}{c_s^2})\label{eq:bvf}
\end{align}

%[inserisci figura freqcat]


\subsection{Stima profondit\'a cavit\'a acustica}

La profondit\'a della cavit\'a acustica varia con il variare della scala orizzontale dell'onda: considero una stratificazione adiabatica

\begin{align*}
    &T=\Dcvar{\TDy{z}{T}}{Ad}\delta&\intertext{$\delta$ \'e la profondit\'a sotto la fotosfera}\\
    &\Dcvar{\TDy{z}{T}}{Ad}=\frac{T}{P}\TDly{P}{T}|_{Ad}\TDy{z}{P}=\frac{\Gamma_2-1}{\Gamma_2}\frac{\mu}{R}g=\frac{g}{c_P}&\intertext{$c_P$ \'e il calore specifico a pressione costante per unit\'a di massa. Scrivendo $c^2=(\Gamma_3-1)g\delta$, da $c=\frac{\omega}{k_h}$ al raggio per cui $k_r=0$ segue:}\\
    &\delta=\frac{\omega^2}{k_h^2(\Gamma_3-1)g}
\end{align*}

I modi con stesso $\frac{\omega}{k_h}$ sono confinati nella stessa cavit\'a.

\subsection{Condizione di risonanza radiale}

In \citet{duv82dispersion} si mostra che in un grafico di $\frac{\pi(n+\alpha)}{\omega}$ in funzione di $\frac{\omega}{k_h}$, i modi p sono rappresentati da un'unica curva per $\alpha$ e n intero opportuni, quindi

\begin{equation}
(n+\alpha)\frac{\pi}{\omega}=F(\frac{\omega}{k_h})\label{eq:duvallr}
\end{equation}


%[Inserisci figura Duvall]

La relazione \eqref{eq:duvallr} \'e dovuta al fatto che, per l fissato, le frequenze dei modi sono determinate dalla condizione che l'onda interferisca costruttivamente con se stessa, si considera la cavit\'a aperta superiormente. un'onda stazionaria in direzione radiale implica che l'integrale di $k_r$ nella regione di propagazione fra due zeri consecutivi sia un intero multiplo di $\pi$.

\begin{align}
&(n+\alpha)\pi\approx\int_{r_t}^Rk_r\,dr\approx\int_{r_t}^R\frac{\omega}{c_S}\sqrt{1-\frac{S_l^2}{\omega^2}}\,dr&\intertext{ho usato la relazione di dispersione per onde acustiche e la frequenza di Lamb \eqref{eq:Lambf}, con}\nonumber\\
&f(w)=\int_{r_t}^R\sqrt{1-\frac{c^2}{w^2r^2}}\,\frac{dr}{c}\label{eq:duvall}
\end{align}


\subsection{Cavit\'a risonanti per modi g}

La regione dei modi g ha come limite superiore N per grandi l, la linea $\omega=\frac{S_lN}{\omega_A}$.

Per i modi g le regioni di propagazione sono quelle per la frequenza \'e minore di entrambi $N$ e $ck_h$.

Le onde di gravit\'a sono presenti nelle regioni in cui il gas \'e neutro o completamente ionizzato ($N^2$ grande) mentre sono riflesse dalle regioni dove $N$ \'e piccolo o immaginario: ionizzazione parziale, instabilit\'a convettiva, centro del Sole.

Ho cavit\'a risonanti per modi g:
\begin{itemize}
    \item Core radiativo.
    
    Tra la la parte centrale dove $g\to0$ e il fondo della zona convettiva dove $N^2<0$.
    \item Atmosfera.
    
    $N$ ha un massimo in coincidenza del punto $T_m$ nella cromosfera: modi g confinati tra zona convettiva e cromosfera ($\Pi\approx\numrange{180}{800}\si{\second}$).
\end{itemize}

\section{Analisi del campo di velocit\'a della superficie solare.}

Una parte basilare dell'informazione contenuta nei modi di oscillazione \'e ricavata analizzando  lo spostamento doppler delle righe di assorbimento dei metalli presenti negli strati visibili pi\'u esterni del sole.
I modi relativi alle oscillazioni dei 5 minuti di grado l non elevato causano uno spostamento quasi totalmente radiale: il segnale \'e proporzionale alla velocit\'a proiettata lungo la linea di vista. Prendendo l'asse delle armoniche sferiche sul piano del cielo ortogonale alla linea di vista, il segnale Doppler osservato \'e proporzionale a

\begin{equation}
    V_D(\theta,\phi,t)=\sin{\theta}\cos{\phi}\sum_{n,l,m}A_{nlm}c_{lm}P_l^m(\cos{\theta})\cos{(m\phi-\omega_{nlm}t-\beta_{nlm})}
\end{equation}

il fattore $\sin{\theta}\cos{\phi}$ deriva dalla proiezione della velocit\'a radiale sulla linea di vista.

Per isolare il contributo di una singola $Y_{l_0m_0}$ considero
\begin{equation}
V_{l_0m_0}(t)=\int_AV_D(\theta,\phi,t)W_{l_0m_0}(\theta,\phi)\,dA=\sum_{n,l,m}S_{l_0m_0,lm}A_{nlm}\cos{(\omega_{nlm}t+\beta_{nlm,L_0m_0})}
\end{equation}
e ho integrato sul disco solare con $W_{l_0m_0}\approx Y_{l_0m_0}$. La funzione di risposta $S_{l_0m_0,lm}\propto\delta_{ll_0}\delta_{mm_0}$ poich\'e le armoniche sferiche sono ortogonali sull'intera sfera $V_{l_0m_0}(t)$ contiene contributi da valori di $(l,m)$ vicini.

La trasformata di Fourier di $V_{l_0m_0}(t)$ permette di isolare i singoli modi caratterizzati dall'ordine radiale n.

Un segnale di durata T permette una risoluzione $\Delta\omega=\frac{2\pi}{T}$: se devo risolvere due frequenze $\omega$ e $\omega+\Delta\omega$ devo osservare per un tempo $T=\frac{2\pi}{\Delta\omega}$ e la frequenza pi\'u bassa osservabile \'e $\Delta\omega$. Il limite superiore delle frequenze osservate \'e dato dalla frequenza di Nyquist $\omega_{Ny}=\frac{\pi}{\Delta t}$ con $\Delta t$ risoluzione temporale, e analogamente per le variabili spaziali e vettore d'onda associato, quindi
\begin{align}
&\Delta\omega=\frac{2\pi}{T}\leq\omega\leq\frac{\pi}{\Delta t}\\
&\Delta k_x=\frac{2\pi}{L_x}\leq k_x\leq\frac{\pi}{\Delta x}
\end{align}


In linea di principio:
\begin{itemize}
    \item Dall'andamento di un modo sulla superficie solare si ricava $(l,m)$.
    \item L'ordine radiale n si ricava dalla distribuzione delle frequenze di oscillazione.
\end{itemize}


\chapter{Modello solare}

La luminosit\'a e la temperatura efficace sono le coordinate nel diagramma di \hr{} di una stella. Tale classificazione consente di dedurre le dimensioni, le caratteristiche del trasporto di energia, la rotazione media e altre caratteristiche di una regione del diagramma.

La massa, l'et\'a e la composizione chimica sono le grandezze di partenza di un modello stellare che riproduca le date L e $T_e$.

La determinazione della struttura solare sulla base delle equazioni fondamentali dell'equilibrio stellare permette anche di calcolare accuratamente le frequenze adiabatiche relative al dato modello solare: dalle discrepanze fra le frequenze osservate e quelle calcolate \'e possibile determinare carenze nella fisica o nelle semplificazioni del modello.

La struttura interna di una stella \'e determinata dalle leggi di conservazione (massa, quantit\'a di moto, energia) e dalle equazioni che caratterizzano lo stato del gas nell'interno solare in funzione di $(T,P)$ .

\section{Conservazione dell'energia.}

La prima legge della termodinamica esprime la conservazione dell'energia interna
\begin{equation*}
\TDy{t}{q}=\TDy{t}{E}+P\TDof{t}(\frac{1}{\rho})=0=\TDy{t}{E}+P\TDy{t}{V}
\end{equation*}

o le equazioni equivalenti

\begin{align}
&\TDy{t}{\ln{T}}=\frac{\Gamma_2-1}{\Gamma_2}\TDy{t}{\ln{P}}+\frac{\TDy{t}{q}}{c_PT}\label{eq:primatemp}\\
&\TDy{t}{\ln{P}}=\Gamma_1\TDy{t}{\ln{\rho}}+\frac{\rho(\Gamma_3-1)}{P}\TDy{t}{q}\label{eq:primapres}
\end{align}

si introducono gli esponenti adiabatici $\Gamma_i$

\begin{equation}
\Gamma_1=\Dcvar{\TDly{\rho}{P}}{Ad}, \ \Gamma_3-1=\Dcvar{\TDly{\rho}{T}}{Ad},\ \frac{\Gamma_2-1}{\Gamma_2}=\Dcvar{\TDly{P}{T}}{Ad}
\end{equation}

Per un gas perfetto, per cui $P\propto\rho T$, si ha $\gamma=\frac{c_P}{c_v}=\Gamma_i$.

Detta $W=E_i+\Omega$, dal teorema del viriale e dalla conservazione dell'energia $\TDy{t}{W}+L=0$ segue

\begin{equation}
L=-\frac{1}{2}\dot{E}_g=\dot{E}_i
\end{equation}

quindi, il tempo caratteristico che regola il collasso gravitazionale di una massa gassosa in equilibrio idrostatico  $\tkh{}=\frac{E_g}{L}\approx\frac{U}{L}\approx\frac{GM^2}{2RL}\approx\SI{1.6e7}{\year}$ determinato dalla velocit\'a del trasporto di energia nelle varie zone.

I processi nucleari che avvengono nella parte centrale forniscono il calore per bilanciare il flusso di energia uscente e mantenere l'equilibrio idrostatico, in particolare le reazioni del ciclo $\Pproton\Pproton$ forniscono attualmente il $99.9\%$ dell'energia

\begin{equation}
\TDy{t}{q}=\epsilon-\frac{1}{\rho}\scap{\nabla}{F}\label{eq:heatgl}
\end{equation}

implica che

\begin{align*}
&\TDy{r}{L}=4\pi r^2[\rho\epsilon-\rho\TDof{t}\frac{u}{\rho}+\frac{P}{\rho}\TDy{t}{\rho}]&\intertext{Nel caso stazionario:}\\
&\TDy{t}{q}=0\ \Rightarrow\ dL=4\pi\rho\epsilon\,dr
\end{align*}

Il tempo trascorso da una stella simile al sole in sequenza principale \'e (fusione tutto H in He) $\tau_n\approx\frac{E_n}{L}=\frac{fX\msun Q}{\lsun}\approx\SI{e+11}{\year}$ con $Q=\SI{6.3e18}{\erg\per\gram}$, f \'e la massa del nucleo di elio che si deve costituire perch\'e la stella evolva dalla sequenza principale rispetto alla massa totale, circa $15\%$ per il Sole.

\section{Trasporto dell'energia.}

I meccanismo di trasporto di energia determinano il gradiente di temperatura stellare: poich\'e nelle regioni interne il cammino libero medio dei fotoni \'e molto corto considero la radiazione localmente in equilibrio con la materia e fissando la luminosit\'a alla superficie si determina il flusso verso l'esterno  e quindi il  gradiente di temperatura, nel caso sia troppo elevato parte della stella \'e soggetta a moti convettivi che trasportano gran parte del flusso verso l'esterno.

La stabilit\'a rispetto a moti convettivi, dovuti a fluttuazioni di densit\'a che generano un moto di materia macroscopico, si realizza se

\begin{align*}
&\Dcvar{\TDy{r}{\rho}}{blob}-\Dcvar{\TDy{r}{\rho}}{amb.}>0&\intertext{assumendo un moto del blob adiabatico e usando l'equazione di stato $\rho(P,T,\mu)$, con $\frac{d\rho}{\rho}=\alpha\frac{dP}{P}-\delta\frac{dT}{T}+\phi\frac{d\mu}{\mu}$, ho il criterio di Ledoux, che caratterizza una regione dinamicamente stabile in cui il trasporto di energia \'e esclusivamente radiativo}\nonumber\\
&\nrad{}<\nad+\frac{\phi}{\delta}\nmu{}\label{eq:ledoux}&\intertext{e ho introdotto i gradienti di temperatura e peso atomico medio definiti da}\nonumber\\
&\nrad{}=\Dcvar{\TDly{P}{T}}{Amb.}=\frac{3}{16\pi acG}\frac{\kappa P}{T^4}\frac{l(r)}{m(r)},\ \nad{}=\Dcvar{\TDly{P}{T}}{Ad}=\frac{P\delta}{T\rho c_P},\ \nmu{}=\Dcvar{\TDly{P}{\mu}}{Amb.}
\end{align*}

La zona convettiva occupa il $30\%$ pi\'u esterno del raggio solare infatti le basse temperature causano un aumento dell'opacit\'a e il gradiente termico necessario per trasportare la luminosit\'a solare \'e superiore al gradiente adiabatico, il cui valore \'e diminuito dal calore latente dell'idrogeno solo parzialmente ionizzato. In questa regione i moti convettivi assicurano l'omogeneit\'a chimica.

Una maggiore efficienza del trasporto convettivo di energia si riflette in una minore differenza tra il gradiente di temperature adiabatico ed effettivo: per determinare lo scostamento dalla stratificazione adiabatica dovuto alle perdite radiative utilizzo la teoria della mixing-length (Select right adiabat???).

Descrivo le caratteristiche del trasporto di energia verso la superficie attraverso la relazione

\begin{equation}
    \TDy{r}{T}=\nabla\frac{T}{p}\TDy{r}{p}
\end{equation}
$\nabla=\TDly{P}{T}$ \'e determinato dalle caratteristiche del trasporto di energia.

\subsection{Teoria della mixing-length}

Per descrivere il trasporto convettivo prendiamo in esame il moto di blob di gas secondo $\TtwoDy{t}{r}=-g\frac{\Delta\rho}{\rho}=g\delta\frac{\Delta T}{T}$ e assumiamo che la pressione sia in equilibrio con l'ambiente.

La lunghezza $l=\alpha H_P$ descrive la distanza percorsa da un blob prima di dissolversi e la velocit\'a media del blob \'e $\exv{v}=l\sqrt{\frac{g\delta}{8H_P}(\nabla-\nabla')}$. Il flusso di energia convettivo \'e $F_C=\alpha\rho c_P\exv{v}T\frac{\nabla-\nabla'}{2}$, dove i gradienti sono quello effettivo e quello riferito al blob, il primato. Il gradiente termico effettivo in presenza di convezione \'e determinato tramite l'equazione
\begin{equation}
\frac{9}{8U}(x-U)^3+x^2-U^2-\nrad{}+\nad{}=0\label{eq:mixingcubic}
\end{equation}
con $x=\sqrt{\nabla-\nad{}+U^2}$.


%[Figura struttura solare zona radiativa zona convettiva]


\section{Costruzione modello del Sole attuale.}

Per risolvere la struttura stellare devo esplicitare la dipendenza dalle variabili $\{P,m,T,L,X_i\}$ di $\rho$, equazione di stato $u(P,T,X_i)$, opacit\'a, $\epsilon(P,T,X_i)$.

\subsection{Equazione di stato??}

Definisco il parametro di plasma per specie s,t:
\begin{align}
&\Lambda_{st}=\frac{3KTr_D}{|e_se_t|}\\
&r_D=\sqrt{\frac{KT}{4\pi\sum_sn_se_s^2}}
\end{align}
che indica il grado di interazione tra le due specie.

\subsection{Energia interna}

L'energia interna per unit\'a di massa \'e determinata dall'energia cinetica delle particelle libere e dall'energia di ionizzazione
\begin{align}
&u(T,P)=\frac{3\gasconstant{}T}{2\mu}+\frac{1}{\rho}[n_{H^+}\chi_H+n_{He^+}\chi_{He}+n_{He^{++}}(\chi_{He}+\chi_{He^+})]
\end{align}

mentre il peso molecolare medio, definito come massa media in amu per particella libera, cio\'e

\begin{equation}
\mu=\frac{1}{\bar{n}_HX+\bar{n}_{He}Y+\bar{n}_{Z}Z}\label{eq:meanmw}
\end{equation}
con $\bar{n}_i=\frac{1+E_i}{A_i}$ numero medio di particelle libere per unit\'a di massa atomica dovute alla specie i di peso atomico $A_i$.

Nel caso del Sole, considerando gli elementi pi\'u pesanti di $^4He$ completamente ionizzati, riscrivo $\mu=\frac{\mu_0}{1+N_E}$, dove $\mu_0$ \'e il peso molecolare medio per H e $^4He$ neutri, e $N_E$ rappresenta il numero di elettroni liberati dalla ionizzazione di H e He diviso il numero delle altre particelle:

\begin{equation}
N_E=\frac{n_e}{n}=\sum_i\nu_i\sum_rx_i^r
\end{equation}
con $X_i$ frazione in massa dell'elemento i con numero atomico $Z_i$, numero di massa $A_i$, peso molecolare $\mu_i$ e $x_i^r$ grado di ionizzazione (numero di atomi del tipo i nello stato di ionizzazione r in unit\'a del numero totale di atomi del tipo i)

\begin{align*}
&\mu_0=\frac{1}{X+Y/4+Z/2}\\
&N_E=\mu_0[\eta_HX+(\eta_{He}+2\eta_{He^+})Y/4]&\intertext{$\eta_A$ indica il grado di ionizzazione della specie A: numero atomi ionizzati sul totale della specie A.}
\end{align*}

\subsection{Reazioni di fusione e processi di diffusione.}

Nella regione radiativa la composizione chimica \'e modificata dalle reazioni di fusione che per gli elementi principali, assumendo condizione di equilibrio secolare, riassumo
\begin{align}
&\dot{X}=\frac{m_p}{N_A}(-3r_{pp}+2r_{33}-r_{34}-4r_{p14})\\
&\dot{Y}_3=\frac{m_{He3}}{N_A}(r_{pp}-2r_{33}-r_{34})\\
&\dot{Y}=\frac{m_{He4}}{N_A}(r_{33}+r_{34}+r_{p14})
\end{align}
con $r_{ik}$ rate di reazione per unit\'a di massa.


L'energia generata per unit\'a di massa \'e
\begin{equation}
\epsilon=\sum Q'_{ik}r_{ik}
\end{equation}
con $Q'_{ij}$ energia liberata per reazione.

Processi di diffusione modificano l'abbondanza degli elementi, il peso molecolare medio e l'opacit\'a. Sebbene il tempo caratteristico per percorre un raggio solare sia lungo $\tau_{diff}\approx\SI{6e13}{\year}$ i processi di diffusione producono effetti misurabili.

I processi di diffusione inglobano diversi effetti: la gravit\'a tende a concentrare gli elementi pi\'u pesanti verso il centro, le interazioni elettromagnetiche mantengono gli elettroni ancorati ai nuclei, la diffusione termica concentra le particelle pi\'u cariche e pi\'u pesanti nelle zone pi\'u calde, mentre la presenza di gradiente di concentrazione $C_s=\frac{n_s}{n_e}$ produce diffusione in senso opposto. Tengo conto del flusso di massa, momento ed energia nelle equazioni di conservazione attraverso le equazioni sviluppate da Burgers (1969): considerando la conservazione della massa, per la concentrazione numerica della specie s ho
\begin{equation}
\PDy{t}{n_s}+\frac{1}{r^2}\PDof{r}(r^2n_sw_s)=\Dcvar{\PDy{t}{n_s}}{Nucl}
\end{equation}
con $w_s$ velocit\'a di diffusione specie s.

\subsection{Parametri del modello e condizioni al bordo.}

Un modello del Sole attuale si ottiene integrando numericamente le equazioni fondamentali della struttura stellare a partire dalle condizioni al bordo
\begin{itemize}
    \item La superficie \'e definata da $T=T_{eff}$ e si ha la condizione $L=4\pi r^2\sigma T^4$. La pressione alla superficie \'e legata alla struttura di equilibrio dell'atmosfera.
    \item In $r=0$ deve essere $L=0$, $M=0$.
    %e condizioni al centro si ricavano espandendo l, m attorno a $r=0$ in termini di $T_c, P_C,X_C,X_{3C}$ ed eguagliando le espansioni ai valori di l e m del punto pi\'u interno.
\end{itemize}

\'E quindi possibile costruire una sequenza evolutiva del Sole in sequenza principale:

l'aumento del peso molecolare medio $\mu$ nella zona di fusione deve essere compensato da un'aumento di temperatura con conseguente incremento dell'energia generata e della luminosit\'a.

L'incertezza sull'abbondanza iniziale di He e sulla profondit\'a della zona convettiva rendono necessaria una calibrazione in funzione della luminosit\'a e raggio attuali. L'et\'a del sistema solare \'e nota grazie modelli di formazione e analisi meteoriti che determinano $t_{\odot}=\SI{4.9+-0.1e9}{\year}$ con incertezza dovuta al periodo di solidificazione dei meteoriti.

La luminosit\'a dipende fortemente dal valore di $Y_0$,  mentre il raggio R dal rapporto fra mixing lenght e scala di pressione $\alpha=\frac{l}{H_P}$, $H_P=-(\TDy{r}{\ln{P}})\expy{-1}$, parametro che regola l'efficienza del trasporto convettivo nella regione esterna.

\begin{align*}
&\ln{L}=\ln{\lsun{}}+a(Y_0-Y_{0\odot})+b(\alpha-\alpha_{\odot})\\
&\ln{r}=\ln{\rsun{}}+c(Y_0-Y_{0\odot})+d(\alpha-\alpha_{\odot})\\
&a=\PDy{Y_0}{\ln{L}}=8.6,\ b=\PDy{\alpha}{\ln{L}}=0.02\\
&c=\PDy{Y_0}{\ln{r}}=2.1,\ d=\PDy{\alpha}{\ln{r}}=-0.19
\end{align*}


Scelgo $Y_0$ e $\alpha$ che forniscono luminosit\'a e raggio $\rsun{}=\SI{6.96e8}{\meter}$ attuali del Sole.

La zona convettiva risulta di \SI{200000}{\kilo\meter} e il valore di $Y_0=0.256$.

L'entropia caratterizza fisicamente la zona convettiva del Sole: l'eccesso di entropia specifica rispetto allo stato marginale
\begin{equation}
    \Delta S=\int c_P(\nabla-\nabla_a)\,d\ln{P}
\end{equation}
diminuisce con $\alpha$.



\part{Oscillazioni lineari adiabatiche.}

In questa sezione ricavo l'equazione del moto perturbato che descrive i modi normali del Sole: considero piccole perturbazioni dello stato di equilibrio e adiabatiche cio\'e molto pi\'u rapide del tempo scala per scambio di calore ( dovuto al flusso radiativo o alle reazioni nucleari). Le oscillazioni adiabatiche dipendono dalla struttura interna del Sole attraverso i coefficienti che compaiono nelle equazioni che descrivono i modi normali. Un confronto con le frequenze osservate necessita che la precisione numerica sia dei parametri ricavati dal modello sia delle tecniche numeriche usate per ricavare le frequenze dei modi di oscillazione sia maggiore dell'accuratezza delle misure.



{\let\clearpage\relax
\chapter{Perturbazioni lineari adiabatiche.}
}

\section{Perturbazione dello stato di equilibrio.}


Descrivo le oscillazioni come piccole perturbazioni attorno allo stato di equilibrio stazionario (gli effetti non lineari sono dell'ordine di $\frac{v}{c_s}$ dove v \'e l'ampiezza dell'oscillazione). Indico con $P'(\vec{r},t)$ la perturbazione euleriana, con $\delta\vec{\xi}$ lo spostamento della particella di fluido a causa della perturbazione e con $\vec{v}=\PDof{t}(\Lvar{\vec{\xi}})$ la sua  velocit\'a

\begin{equation}
P(\vec{r},t)=P_0(\vec{r})+P'(\vec{r},t)\label{eq:pressureperturbation}
\end{equation}

e in termini della variazione lagrangiana

\begin{equation*}
\Lvar{P(\vec{r})}=P(\vec{r}+\Lvar{\vec{\xi}})-P_0(\vec{r})=P'(\vec{r})+\Lvar{\vec{\xi}}\cdot\nabla P_0
\end{equation*}



Ricavo l'equazione del moto perturbato sostituendo \eqref{eq:pressureperturbation} nell'equazione del moto  $\rho\TDof{t}v\indices{_i}=\rho(\PDy{t}{v\indices{_i}}+v\indices{_j}\partial\indices{_j}v\indices{_i})=-\partial\indices{_i} P+\rho\vec{g}\indices{_i}$ e considerando solo i termini lineari nelle perturbazioni risulta:

\begin{equation}
\rho_0\PtwoDy{t}{\Lvar{\vec{\xi}}}=\rho_0\PDy{t}{\vec{v}}=-\nabla P'+\rho_0\vec{g}'+\rho'\vec{g}_0\label{eq:emper}
\end{equation}

con $\vec{g}'=-\nabla\Phi'$, $\nabla^2\Phi'=4\pi G\rho'$.

Analogamente per l'equazione di continuit\'a ottengo
\begin{equation}
\rho'+\div{(\rho_0\Lvar{\vec{\xi}})}=0\label{eq:contper}
\end{equation}


\subsection{Approssimazione adiabatica}

I tempi caratteristici per scambio di calore sono molto maggiori del periodo delle pulsazioni. Considero i termini a destra dell'equazione di conservazione dell'energia interna \eqref{eq:primatemp}, dove ho esplicitato il bilancio di calore usando \eqref{eq:heatgl},


\begin{equation*}
\TDy{t}{T}-\frac{\Gamma_2-1}{\Gamma_2}\frac{T}{P}\TDy{t}{P}=\frac{1}{c_P}(\epsilon-\frac{1}{\rho}\scap{\nabla}{F})
\end{equation*}

Stimo il tempo caratteristico $\tau_R$ per flusso radiativo \eqref{eq:radiativeflux}:

\begin{align*}
&\frac{1}{\rho c_P}\nabla\cdot(\frac{4acT^3}{3\kappa\rho}\nabla T)\approx\frac{4acT^4}{3\kappa\rho^2c_PH}=\frac{T}{\tau_R}&\intertext{H lunghezza caratteristica, in cgs:}\\
&\tau_R=\num{e12}\frac{\kappa\rho^2H^2}{T^3}
\end{align*}

Per valori caratteristici solari ($\kappa=1$, $\rho=1$, $T=\num{e6}$, $H=\num{e10}$) ho $\tau_R\approx\SI{e7}{\year}\approx\tkh{}$, per valori caratteristici della zona convettiva ($\kappa=100$, $\rho=\num{e-5}$, $T=\num{e4}$, $H=\num{e9}$) ho $\tau_R\approx\SI{e3}{\year}$.


Nella parte interna il termine dovuto alle reazioni nucleari \'e caratterizzato da un tempo $\tau_{\epsilon}\approx\frac{c_PT}{\epsilon}\approx\tkh{}$.

Confronto $\frac{T}{\tau_R}$, $\frac{T}{\tau_{\epsilon}}$ con $\TDy{t}{T}\approx\frac{T}{\Pi_{osc}}$ con $\Pi_{osc}\approx\si{\minute}-\si{\hour}$: i termini dovuti allo scambio di calore sono trascurabili rispetto alla derivata temporale di T.

Il moto di una elemento di fluido \'e descritto dalla relazione adiabatica


\begin{align*}
&\TDy{t}{P}=\frac{\Gamma_1P}{\rho}\TDy{t}{\rho}
\end{align*}

Approssimazione adiabatica non pi\'u valida vicino alla superficie solare dove i tempi per lo scambio di calore sono pi\'u brevi.

La condizione di perturbazione adiabatica linearizzata \'e
\begin{align}
&\PDy{t}{\Lvar{P}}-\frac{\Gamma_{1,0}P_0}{\rho_0}\PDy{t}{\Lvar{\rho}}=0\nonumber&\intertext{che integrata rispetto a t ed in funzione della variazione euleriana diventa}\nonumber\\
&P'+\Lvar{\vec{\xi}}\cdot\nabla P_0=\frac{\Gamma_{1,0}P_0}{\rho}(\rho'+\Lvar{\vec{\xi}}\cdot\nabla\rho_0)\label{eq:adper}
\end{align}

\section{Modi di oscillazione.}

\subsection{Separazione variabili spaziali e temporali.}

Dall'equazione del moto \eqref{eq:emper} si vede che

\begin{equation*}
\hat{r}\cdot(\rot{\PtwoDy{t}{\vec{\xi}}})=0\ \Rightarrow\ \PDof{\theta}(\sin{\theta}\xi_{\phi})-\PDy{\phi}{\xi_{\theta}}=0
\end{equation*}

quindi \'e possibile ricavare la componente tangenziale della perturbazione da una funzione scalare.

Descrivo i modi normali di oscillazione tramite la dipendenza temporale da $\exp{i\omega t}$ e angolare tramite le funzioni armoniche sferiche $Y_{lm}(\theta,\phi)$

\begin{equation}
\vec{\xi}=\exp{i\omega t}(\xi_r(r),\xi_h(r)\PDof{\theta},\frac{\xi_h(r)}{\sin{\theta}}\PDof{\phi})Y_l^m(\theta,\phi)
\end{equation}

le funzioni armoniche sferiche  soddisfano

\begin{equation}
L^2Y_l^m=-\frac{1}{\sin{\theta}}\PDof{\theta}(\sin{\theta}\PDy{\theta}{Y_l^m})+\frac{1}{\sin^2{\theta}}\PtwoDy{\phi}{Y_l^m}=-r^2\nabla_h^2Y_l^m=l(l+1)Y_l^m
\end{equation}

La variazione euleriana di densit\'a, pressione, potenziale gravitazionale sono espresse
\begin{align*}
&(\rho',P',\Phi')=\exp{i\omega t}[\rho'(r),P'(r),\Phi'(r)]Y_l^m
\end{align*}


\subsection{Determinazione dello spettro delle oscillazioni.}

Utilizzo l'equzione del moto \eqref{eq:emper} e l'equazione di continuit\'a \eqref{eq:contper} per eliminare $\xi_h(r)$ dall'equazione del moto
\begin{align}
&\frac{1}{r^2}\TDof{r}(r^2\xi_r)-\frac{\xi_rg}{c^2}+\frac{1}{\rho_0}(\frac{1}{c^2}-\frac{l(l+1)}{r^2\omega^2})P'-\frac{l(l+1)}{r^2\omega^2}\Phi'=0\nonumber\\
&\frac{1}{\rho_0}(\TDof{r}+\frac{g}{c^2})P'-(\omega^2-N^2)\xi_r+\TDy{r}{\Phi'}=0\label{eq:eigenomega}\\
&\frac{1}{r^2}\TDof{r}(r^2\TDy{r}{\Phi'})-\frac{l(l+1)}{r^2}\Phi'-\frac{4\pi G\rho_0}{g}N^2\xi_r-\frac{4\pi G}{c^2}P'=0\nonumber
\end{align}

(con $g=-\frac{1}{\rho_0}\TDy{r}{P_0},\ c^2=\frac{\Gamma_1P_0}{\rho_0}$)

Ho definito la frequenza di \bv{}

\begin{equation*}
N^2=g(\frac{1}{\Gamma_1P_0}\TDy{r}{P_0}-\frac{1}{\rho_0}\TDy{r}{\rho_0})
\end{equation*}

che definisce la massima frequenza sotto cui pu\'o oscillare una particella di fluido sottoposta a onde di gravit\'a, e la frequenza di Lamb $S_l^2=\frac{l(l+1)c^2}{r^2}$ che determina il punto in cui la propagazione di un'onda acustica \'e orizzontale: come spiego meglio nella prossima parte, definiscono le regioni di propagazione dei modi.

Il sistema di equazione ~\eqref{eq:eigenomega} ha soluzione con le opportune equazioni al contorno per un insieme discreto di valori delle frequenze $\omega_{nlm}$, l'ordine angolare non compare nelle equazioni quindi gli autovalori $\omega_{nlm}$ sono $2l+1$ degeneri. La degenerazione \'e rimossa nel caso si tenga conto della rotazione ($\frac{\Omega}{\omega}\approx\num{e-4}$) o di effetti gravitazionali di altri corpi.


Condizioni al contorno: sono necessarie 4 condizioni.

\begin{itemize}
\item Due condizioni per $r=0$ selezionano le soluzioni regolari:

\begin{equation}
P'=0,\ \Phi'=0
\end{equation}

Vicino a zero risulta un andamento asintotico

\begin{align*}
&(l\neq0):\ \xi_r\propto r\expy{l-1};\ (l=0):\ \xi_r\propto r\\
&P',\ \Phi'\propto r^l
\end{align*}

\item Alla superficie solare richiediamo la continuit\'a di $\Lvar{\nabla\Phi}$ e che non si abbia propagazione verso l'esterno.

All'esterno della stella ho $\rho'=0$ quindi scelgo la soluzione nulla a infinito dell'equazione di Poisson $\Phi'=Ar\expy{-l-1}$:

\begin{equation}
\TDy{r}{\Phi'}+\frac{l+1}{r}\Phi'=0,\ r=\rsun{}    
\end{equation}

La condizione di non propagazione altre la fotosfera dipende dalla descrizione dell'atmosfera solare. Nella versione pi\'u semplice impongo che la pressione sia zero alla superficie perturbata della stella

\begin{align}
&\Lvar{P}=P'+\xi_r\TDy{r}{P}=0
\end{align}

\end{itemize}


%\section{Stabilit\'a dei modi di oscillazione.}



\chapter{Caratteristiche asintotiche delle oscillazioni adiabatiche.}

%[inserisci figura propagation ag]

Descrivo il comportamento asintotico delle oscillazioni: nel limite per basse frequenze ho i modi g per alte frequenze i modi p e f. Ricavo le frequenze critiche per i modi gravo-acustici che definiscono le regioni di propagazione. Nell'approssimazione di onde piane si ricava una espressione approssimata per le frequenze dei modi di oscillazione.

Quando le frequenza sono molto grandi, per i modi p, o molto piccole, per i modi g, \'e possibile ricavare soluzioni analitiche approssimate delle equazioni delle oscillazioni ed esprimere analiticamente la dipendenza delle frequenze dei modi di oscillazione dai parametri del modello solare.

\section{Comportamento asintotico.}

Per determinare la struttura dello spettro delle oscillazioni introduciamo l'approssimazione di Cowling (\cite{cow41oscillations}) cio\'e trascuriamo la perturbazione del potenziale gravitazionale. Quindi il sistema \eqref{eq:eigenomega} si riduce al secondo ordine

\begin{align}
&\frac{1}{r^2}\TDof{r}(r^2\xi_r)-\frac{\xi_rg}{c^2}+\frac{1}{\rho_0}(\frac{1}{c^2}-\frac{l(l+1)}{r^2\omega^2})P'=0\label{eq:cowosc}\\
&\frac{1}{\rho_0}(\TDof{r}+\frac{g}{c^2})P'-(\omega^2-N^2)\xi_r=0\nonumber
\end{align}

Considero i limiti asintotici di alte e basse frequenze: ottengo un problema del tipo di Sturm-Liuville.
%https://en.wikipedia.org/wiki/Sturm%E2%80%93Liouville_theory

\begin{itemize}
\item Per $\omega\to\infty$:

Lo spettro \'e discreto, le oscillazioni sono prodotte da onde acustiche in cui la forza dominante \'e fornita dalla pressione, chiamati modi p, ordinati in base al numero di zeri di $\xi_r$ fra il centro e la superficie. Considerazioni stabilit\'a.

\item Per $\omega\to0$:

Lo spettro \'e discreto, il moto \'e determinato dalla forza di gravit\'a, sono chiamati quindi modi g, ordinati secondo il numero di nodi radiali.

La stabilit\'a dei modi g \'e determinata dalla stabilit\'a convettiva: se non sono presenti regioni di instabilit\'a convettiva i modi g sono stabili ($g_+$), se esistono zone convettivamente instabili esistono anche modi g instabili ($g_-$).

\end{itemize}

Lo spettro solare \'e la combinazione dei modi parziali precedenti; il modo f separa  i modi g e p: non ha nodi in direzione radiale.


\subsection{Relazione di dispersione per i modi gravo-acustici.}

Approssimo il comportamento spaziale delle oscillazioni con quello di onda piana
\begin{align*}
&\vec{\xi}\propto\exp{i\scap{k}{x}},\ \vec{k}=k_r\hat{r}+\vec{k}_h\\
&S_l^2=\frac{l(l+1)c^2}{r^2}\approx k_h^2c^2
\end{align*}

Considero i coefficienti delle equazioni \eqref{eq:cowosc} costanti ad eccezione di $P_0,\ \rho_0$: approssimazione valida se la lunghezza d'onda delle perturbazioni \'e molto minore della scala caratteristica di variazione dei coefficienti.

%stix local treatment pg 165


Sostituisco
\begin{align}
&\xi_r\propto\rho_0\expy{-\frac{1}{2}}\exp{ik_rr},\ P_1\propto\rho_0\expy{\frac{1}{2}}\exp{ik_rr}&\intertext{forma richiesta dalla conservazione dell'energia e della quantit\'a di moto, e ottengo la relazione di dispersione:}\nonumber\\
&k_r^2=\frac{\omega^2-\omega_A^2}{c^2}+S_l\frac{N^2-\omega^2}{c^2\omega^2}=\frac{\omega^2}{c^2}(1-\frac{\omega_{l,+}^2}{\omega^2})(1-\frac{\omega_{l,-}^2}{\omega^2})\label{eq:localdispersion}&\intertext{e quindi determino le frequenze critiche per i modi gravo-acustici}\\
&\omega_{\pm}=\frac{1}{2}(S_l^2+\omega_c^2)\pm\sqrt{\frac{1}{2}(S_l^2+\omega_c^2)^2-N^2S_l^2}
\end{align}


\section{Asymptotic properties of p modes}

Posso trascurare N e, eccetto vicino alla superficie, $\omega_c\ll\omega$, mentr vicino alla superficie $S_l\ll\omega$ for small/moderate l.

\begin{align*}
&\omega\int_{r_1}^{r_2}\sqrt{1-\frac{\omega_c^2}{\omega^2}-\frac{S_l^2}{\omega^2}}\,\frac{dr}{c}\approx\pi(n-\frac{1}{2})&\intertext{where $r_1=r_t$, $r_2=R_t$. With help of our assumption we can expand the integral and, introducing the function $\alpha(\omega)$ depending only on frequency and near surface behaviour of $\omega_c$.}
\end{align*}

For low degree modes we use the fact that integrand differs from 1 only close to lower turning point close to center for low order mode ($F(w)\approx\int_0^R\frac{dr}{c}-w\expy{-1}\frac{\pi}{2}$)

\begin{align*}
&\nu_{nl}=\frac{\omega_{nl}}{2\pi}\approx(n+\frac{l}{2}+\frac{1}{4}+\alpha)\Delta\nu\\
&\Delta\nu=[2\int_0^R\frac{dr}{c}]\expy{-1}&\intertext{is the inverse of twice travel time center/surface. This equation predict uniform spacing in n of frequency of low degree modes (claverie79)}
\end{align*}

Deviazioni da questa legge hanno potenziale diagnostico per la parte interna, infatti estendendo l'espansione di

\begin{equation*}
F(w)=\int_{r_t}^R\sqrt{1-\frac{c^2}{w^2r^2}}\,\frac{dr}{c}
\end{equation*}

fino al termine dipendente dalla variazione di c:

\begin{align*}
d_{nl}=\nu_{nl}-\nu_{n-1,l+2}\approx-(4l+6)\frac{\Delta\nu}{4\pi^2\nu_{nl}}\int_0^R\TDy{r}{c}\,\frac{dr}{c}&\intertext{sound speed is reduced as $\mu$ increases with H to He conversion as star ages: as a result $d_{nl}$ is reduced providing measure of evolutionary state of stars}
\end{align*}


\subsection{Asymptotic g modes}

In inner domain an expansion in terms of $\frac{\omega^2}{S_l^2}$ is possible, while $\frac{\omega^2}{N^2}$ serves as small expansion parameter in outer domain containing the surface, additional domains have to be considered for zeros of $N^2$

For the Sun we have $N^2(r_v)=0$ where $r_v$ marks lower bound of convection zone, matching the respective expansion we have in first order
\begin{align*}
&T_{n,l}=\frac{2\pi^2(n+\frac{l}{2}-\frac{1}{4})}{\sqrt{l(l+1)}}(\int_0^{r_v}\frac{N}{r}\,dr)\expy{-1}=\frac{n+\frac{l}{2}-\frac{1}{4}}{\sqrt{l(l+1)}}T_0
\end{align*}

g modes have equidistant period spacing.



\begin{comment}

\chapter{Regioni di propagazione: cavit\'a risonanti.}\label{chap:propagationr}

%[Inserisci figura khomeagisot]
%[Inserisci figura pgmodesC]


\section{Regioni di propagazione: cavit\'a risonanti.} %% Trasferito dopo prima dell'introduzione della relazione di dispersione

Nel Sole sono presenti regioni in cui un'onda di data frequenza si pu\'o propagare limitate da regioni in cui non si ha propagazione. Abbiamo due tipi di  la variazione delle condizioni del gas determina 

\'E possibile analizzare tramite metodo  JWKB il sistema di equazioni delle oscillazioni del secondo ordine in approssimazione di Cowling, previa oppurtuna trasformazione, da cui si ottiene la relazione
\begin{equation}\label{eq:jwkb}
\omega\int_{r_1}^{r_2}[1-\frac{\omega_A^2}{\omega^2}-\frac{S_l^2}{\omega^2}(1-\frac{N^2}{\omega^2})]\expy{\frac{1}{2}}\frac{dr}{c}\approx\pi(n-\frac{1}{2})
\end{equation}
dove $r_1$ e $r_2$ sono due zeri consecutivi del numero d'onda radiale e l'integrazione \'e in una regione di propagazione.

Nel caso dei modi p e assumendo $S_l\ll\omega$ vicino al punto di inversione superiore ho

\begin{equation}\label{eq:jwkbmodep}
\omega\int_{r_1}^{r_2}[1-\frac{S_l^2}{\omega^2}]\expy{\frac{1}{2}}\frac{dr}{c}\approx\pi(n-\alpha{\omega})
\end{equation}



\section{Cavit\'a acustiche.}       %% Trasferito prima di asimptotic properties of p modes
Per grandi $\omega$ ~\eqref{eq:localdispersion} si riduce alla relazione di dispersione acustica 

\begin{equation*}
\omega^2=c^2(k_r^2+k_h^2)
\end{equation*}

Posso ricavare il raggio di inversione del moto in direzione radiale $k_r=0$ dalla relazione di dispersione per onde onde acustiche, da cui segue
\begin{equation}
\frac{c(r_i)}{r_i}=\frac{\omega}{l(l+1)}
\end{equation}

Maggiore \'e il grado l (piccolo $\lambda_h$) meno profonda \'e la cavit\'a: sono riflesse verso la superfice quando la velocit\'a del suono \'e aumentata fino alla loro velocit\'a di fase orizzontale; la profondit\'a della cavit\'a acustica varia con il variare della scala orizzontale dell'onda. (Top  convection zone down to the level at which refraction due to sound speed increasing $c\propto\sqrt{T}$ turn the wave around when $c=\frac{\omega}{k_h}$)

Stima profondit\'a cavit\'a acustica
\begin{align*}
    &T=\Dcvar{\TDy{z}{T}}{Ad}\delta&\intertext{$\delta$ \'e la profondit\'a sotto la fotosfera}\\
    &T=\Dcvar{\TDy{z}{T}}{Ad}=\frac{T}{P}\TDly{P}{T}|_{Ad}\TDy{z}{P}=\frac{\gamma-1}{\gamma R}g=\frac{g}{c_P}\\
    &c^2=(\gamma-1)g\delta&\intertext{da $c=\frac{\omega}{k_h}$ segue:}\\
    &\delta=\frac{\omega^2}{k_h^2(\gamma-1)g}
\end{align*}
minore la lunghezza d'onda orizzontale pi\'u sottile la cavit\'a.

Vicino alla superficie l'efficienza della convezione diminuisce, il gradiente di temperatura diventa fortemente sopra-adiabatico e la fraquenza critica $\omega_A$ aumenta notevolmente: le onde acustiche con periodo attorno ai 5-min diventano evanescenti in poche scale di altezza: l'inizio della zona convettiva \'e uno specchio a larga banda per onde acustiche. 

%Duvall82 (....----)

\subsection{Legge di Duvall}

In un grafico $\frac{\omega}{k_h}$ vs $\frac{\pi(n+\alpha)}{\omega}$ i modi p sono rappresentati da un'unica curva. Se considero la differenza di fase
\begin{align}\label{eq:duvall}
&\Delta\phi=\int_{r_t}^{\rsun{}}k_r\,dr=\int_{r_t}^{\rsun{}}(\frac{1}{c^2}-\frac{l(l+1)}{r^2\omega^2})\expy{\frac{1}{2}}\,dr\\
&=F(\frac{\omega}{L})\\
&\Delta\phi=\pi(n+\alpha)
\end{align}
tra i bordi interno ed esterno della cavit\'a acustica per un modo di oscillazione $\Delta\phi=\pi(n+\alpha)$ la costante $\alpha$ \'e necessaria dato che i bordi non sono rigidi.
L'integrale risulta funzione di $\frac{\omega}{k_h}$. 

[Inserisci figura Duvall]



\section{Cavit\'a risonanti per modi g.}        %% Trasferito tra Asymptotic properties of p modes e asimptotic g

Nella parte a basse frequenze dei modi g la relazione \ref{eq:localdispersion} si approssima, per $l\neq0$ con

\begin{equation*}
k_r^2=\frac{S_l^2}{c^2}(\frac{N^2}{\omega^2}-1)
\end{equation*}

La regione dei modi g ha come limite superiore N per grandi l, la linea $\omega=\frac{S_lN}{\omega_A}$.

Per i modi g le regioni di propagazione sono quelle per la frequenza \'e minore di entrambi $N$ e $ck_h$.

La struttura degli strati esterni del sole \'e dominata dalla ionizzazione di H e He con conseguente aumento dell'opacit\'a e quindi del gradiente di temperatura in equilibrio radiativo e il calore specifico: il gradiente di temperatura critico per instabilit\'a convettiva $\frac{g}{c_P}$ diminuisce. In questa regione il gradiente di temperatura \'e debolmente super-adiabatico, $N^2<0$: la zona convettiva costituisce una barriera per le onde di gravit\'a interne.

Le onde di gravit\'a sono presenti nelle regioni in cui il gas \'e neutro o completamente ionizzato ($N^2$ grande) e sono riflesse in regioni dove $N$ \'e piccolo o immaginario: ionizzazione parziale, instabilit\'a convettiva, centro del Sole.

Ho cavit\'a risonanti per modi g:
\begin{itemize}
    \item Core radiativo.
    
    Tra la la parte centrale dove $g\to0$ e il fondo della zona convettiva dove $N^2<0$.
    \item Atmosfera.
    
    $N$ ha un massimo in coincidenza del punto $T_m$ nella cromosfera: modi g confinati tra zona convettiva e cromosfera ($\Pi\approx\numrange{180}{800}\si{\second}$).
\end{itemize}

\end{comment}


\part{Problema inverso: correzione al modello dalle oscillazioni.}


\chapter{Tecniche asintotiche.}

Per modi di basso grodo \'e possibile espandere al primo ordine l'integrale nella \eqref{eq:duvall} $F(w)\approx\int_0^R\frac{dr}{c}-w\expy{-1}\frac{\pi}{2}$ ed esprimere la legge di Duvall tramite
\begin{equation}\label{eq:claverie}
    \nu_{nl}=\frac{\omega_{nl}}{2\pi}\approx(n+\frac{l}{2}+\frac{1}{4}+\alpha)\Delta\nu
\end{equation}
con $\Delta\nu=[\int_0^R\frac{dr}{c}]\expy{-1}$.
La presenza di picchi uniformemente spaziati di modi a basso grado l \'e stata osservata da Cleverie79.

Estendendo ancora l'espansione di \eqref{eq:duvall} si ha una misura della variazione di $c$ nel core della stella
\begin{equation}\label{eq:tassoul}
    d_{nl}=\nu_{nl}-\nu_{n-1,l+2}\approx-(4l+6)\frac{\Delta\nu}{4\pi^2\nu_{nl}}\int_0^R\frac{dc}{dr}\frac{dr}{r}
\end{equation}
La velocit\'a del suono \'e ridotta a causa dell'aumentare di $\mu$ durante la fusione di H in He durante l'evoluzione stellare e quindi $d_{nl}$ \'e ridotto.


\chapter{Linearizzazione della ''variazione'' attorno ad un modello solare.}

\section{Principio variazionale}

Riscrivo l'equazione del moto linearizzata nella forma
\begin{equation}
    \omega^2\Lvar{\vec{r}}=\frac{1}{\rho}\nabla p'-\vec{g}'-\frac{\rho'}{\rho}\vec{g}=\mathcal{F}(\Lvar{\vec{r}})
\end{equation}
da cui risulta che in seguito ad una perturbazione del modello di equilibrio $\Lvar{\mathcal{F}}$ le frequenze delle oscillazioni adiabatiche sono determinate da 

\begin{equation}\label{eq:variational}
    \Lvar{\omega^2}=\frac{\int_V\Lvar{\vec{r}}^*\cdot\mathcal{F}(\Lvar{\vec{r}})\rho\,dV}{\int_V|\Lvar{\vec{r}}|^2\rho\,dV}
\end{equation}
$\Lvar{\vec{r}}$ \'e autovalore per il problema imperturbato.




\section{Rotazione.}

Il Sole \'e un rotatore lento.



We want to find a velocity field which in spherical coordinates has the form
\begin{align*}
&\vec{v_0}=(0,0,r\Omega\sin{\theta})=\vecp{\Omega}{r}\\
&\vec{\Omega(r,\theta)}=(\Omega(r,\theta)\cos{\theta},-\Omega(r,\theta)\sin{\theta},0)&\intertext{il vettore velocit\'a angolare \'e funzione di r e $\theta$}
\end{align*}

Without rotation the inertia term is $\rho_0\TDy{t}{\vec{v}}=\rho_0\PtwoDy{t}{\vec{\xi}}$ where there is no mean motion, in case of rotation $\rho_0(\PDof{t}+\scap{v_0}{\nabla})^2\vec{\xi}$.

We consider additional term as a small perturbation

\begin{align*}
&\PDof{t}=i\omega\\
&\omega=\omega_{\alpha}+\Delta\omega_{\alpha}\\
&Y_{\alpha}=Y_{lm}\\
&\rho_0(\omega_{\alpha}^2+2\omega_{\alpha}\Delta\omega_{\alpha})\vec{\xi}\\
&=\nabla P_1-\frac{\rho_1}{\rho_0}\nabla P_0+\rho_0\nabla\Phi_1+2i\omega_{\alpha}\rho_0(\scap{v_0}{\nabla})\vec{\xi}&\intertext{equazione del moto al primo ordine nella perturbazione}
\end{align*}

quindi risulta

\begin{align*}
&\Delta\omega_{\alpha}=\frac{i\int\rho_0\xi_{\alpha}^*(\scap{v_0}{\nabla})\xi_{\alpha}}{\int\rho_0\xi_{\alpha}^*\xi_{\alpha}}\\
&\Delta\omega_{\alpha}=\frac{-m\int\rho_0\Omega\xi_{\alpha}^*\xi_{\alpha}\,dV+i\int\rho_0\xi_{\alpha}^*(\vecp{\Omega}{\xi_{\alpha}})\,dV}{\int\rho_0\xi_{\alpha}^*\xi_{\alpha}}\intertext{usando $\vec{v_0}=\vecp{\Omega}{r}$}
\end{align*}

Dobbiamo trovare $\Omega(r,\theta)$ dalla differenza $\Delta\omega_{\alpha}$: the problem is linear in $\Omega$ so the shift $\Delta\Omega_{\alpha}$ is of the same order as $\Omega$.

For evaluation of shift formula we must know eigenfunction $\xi_{\alpha}$ of unperturbed state.

Per rotazione puramente radiale $\Omega(r)$
\begin{align*}
&\Delta\omega_{\alpha}=-m\frac{\int_0^{\rsun{}}\rho_0\Omega\{|\xi_r-\xi_h|^2+[l(l+1)-2]|\xi_h|^2\}r^2\,dr}{\int_0^{\rsun{}}\rho_0\{|\xi_r|^2+l(l+1)|\xi_h|^2\}r^2\,dr}\\
&=\int_0^{\rsun{}}K_{\alpha}(r)\Omega(r)\,dr
\end{align*}

nel caso di rotazione dipendente solo da r $\Delta\omega_{\alpha}$ \'e lineare in m, $2l+1$ frequencies with equidistant spacing.

Any given $\Delta\Omega_{\alpha}$ samples angular velocity in the depth range corresponding to $\xi_{\alpha}$.

Le osservazioni della superficie mostrano una dipendenza dalla co-latitudine 

\begin{equation*}
\frac{\Omega(\theta)}{2\pi}=\SI{451.5}{\nano\hertz}-\SI{65.3}{\nano\hertz}\cos^2{\theta}-\SI{66.7}{\nano\hertz}\cos^4{\theta}
\end{equation*}

risultato di un best fit (discrepanze notevoli e variazioni temporali).

For an investigation of the full function $\Omega(r,\theta)$ the whole multiplet $2l+1$ frequencies must be used: deviation from equidistant spacing within the multiplet is typical of latitudinal shear.


\section{Inversione non asintotica.}

In the structure case the relation between structure and multiplet frequencies is highly nonlinear: we perform linearization on the assumption that a solar model close enough to actual solar structure exists.

\subsection{Correzioni struttra idrostatica}

\'E possibile quindi mettere in relazione la differenze tra le frequenze osservate  con quelle calcolate da un modello, $\delta\omega_{nl}=\Omega_{\odot}-\Omega_{Mod}$ e le differenze nella stratificazione idrostatica

\begin{align}
&\frac{\delta\omega_{nl}}{\omega_{nl}}=\int_0^R[K^{nl}_{c^2,\rho}(r)\frac{\delta_rc^2}{c^2}(r)+K^{nl}_{\rho,c^2}(r)\frac{\delta_r\rho}{\rho}(r)]\,dr\\
&+I_{nl}\expy{-1}F_{Surf}(\omega_{nl})\\
&\frac{\delta_rc^2}{c^2}(r)=\frac{[c_{\odot}^2(r)-c_{mod}^2(r)]}{c^2(r)}\\
&\frac{\delta_r\rho}{\rho}(r)=\frac{[\rho_{\odot}(r)-\rho_{mod}(r)]}{\rho(r)}\label{eq:invstructure}
\end{align}

i kernel $K_Q^j$ dipendono dalle autofunzioni del modello, il termine $I_{nl}\expy{-1}F_{Surf}(\omega_{nl})$, $I_{nl}=\int_V|\Lvar{\vec{r}}|^2\rho\,dV$ \'e una correzione dovuta alle differenti condizioni fisiche che si incontrano vicino alla superficie: per basse frequenze si ha riflessione pi\'u in profondit\'a a $\omega=\omega_c$ e quindi risentono meno degli effetti degli strati superficiali.

The analysis in terms of $\frac{\delta_rc^2}{c^2}(r)$ and $\frac{\delta_r\rho}{\rho}(r)$ capture the difference between Sun and model related hydrostatic structure.

\subsection{Correzioni equazione di stato e composizione.}

Since sound speed depends upon $\Gamma_1$ as $c_s^2=\Gamma_1\frac{P}{\rho}$ we can express $\Gamma_1(P,\rho,Y,Z)$ from thermodynamic properties and composition of the gas.

We obtain equivalent formulation of \ref{eq:invstructure} expressing $\delta_rc^2$ in terms of $\delta_rP$, $\delta_r\rho$, $\delta_rY$ and $\delta_r\Gamma_1$

\begin{align}
&\frac{\delta\omega_{nl}}{\omega_{nl}}=\int_0^RK^{nl}_{u,Y}(r)\frac{\delta_ru}{u}(r)\,dr+\int K^{nl}_{Y,u}(r)\delta_rY\,dr\\
&+\int_0^RK^{nl}_{c^2,\rho}(r)(\frac{\delta\Gamma_1}{\Gamma_1})_{int}\,dr+I_{nl}\expy{-1}F_{Surf}(\omega_{nl})\label{eq:diffthermo}&\intertext{allowance in error $(\delta\Gamma_1)_{int}$, difference between $\Gamma_1$ Sun and $\Gamma_1$ model EOS.}
\end{align}

Fatti:
\begin{itemize}
    \item Inversione di $\Gamma_1$ mostra la necessit\'a di tener conto degli effetti relativistici per gli elettroni (average thermal energy approx \SI{1.35}{\kilo\ev} approx $0.3\%$ of $m_e$).
\end{itemize}


\chapter{(Numerical) Inversion technique.}

\subsection{Least square inversion.}

Parametrize unknown functions $\frac{\delta_rc^2}{c^2}$, $\frac{\delta_r\rho}{\rho}$, $F_{Surf}$ (Slowly variable polynomials), the parameter being determined through regularized least square fitting (Dziembowski90, Antia Basu 94).

\subsubsection{(Regularized least square methods)}
(JCD90).

\subsection{(OLA)}
(Backus, Gibbert 68,70; Gough 85).

%For rotation inversion $\omega=\omega_{0nl}+m\omega_{1nl}$, $\omega_{1nl}=\int_0^1K_{nl}(x)\Omega(x)\,dx+\epsilon_{nl}$ con $x=\frac{r}{R}$ e $\epsilon_{nl}$ errori in $\omega_{1nl}$
%$\ensemble{c_i(r_0)}$

\subsection{SOLA}
(Pijpers, thompson 92).

Let's determine $\frac{\delta_rc^2}{c^2}$

Expression to be minimized

\begin{align*}
&\int_0^R[\mathcal{K}_{c^2,\rho}(r_0,r)-\mathcal{T}(r_0,r)]^2\,dr\\
&+\beta\int_0^R\mathcal{G}_{\rho,c^2}(r_0,r)\,dr+\mu\sum_i\sigma_ic_i(r_0)c_j(r_0)\\
&\mathcal{K}_{c^2,\rho}(r_0,r)=\sum_ic_i(r_0)K_{c^2,\rho}^i(r)&\intertext{averaging kernel}\\
&\mathcal{G}_{\rho,c^2}(r_0,r)=\sum_ic_i(r_0)K_{\rho,c^2}^i(r)&\intertext{cross-term kernel which controls the undesidered contrib from $\frac{\delta_r\rho}{\rho}$}
\end{align*}

dove $i=(n,l)$ e $\sigma_i$ \'e l'errore su $\frac{\delta\omega_i}{\omega_i}$.

In general we choose coefficient $c_i(r_0)$ such that $\sum c_i(r_0)\frac{\delta\omega_i}{\omega_i}$ provides a localized average of $\frac{\delta f_1(r)}{f_1(r)}$ around $r=r_0$:

\begin{align*}
&\sum_ic_i(r_0)\frac{\delta\omega_i}{\omega_i}=\int_0^R\sum_ic_i(r_0)K_{1,2}^i(r)\frac{\delta f_1(r)}{f_1(r)}\,dr\\
&+\int_0^R\sum_ic_i(r_0)K_{2,1}^i(r)\frac{\delta f_2(r)}{f_2(r)}\,dr\\
&+\sum_ic_i(r_0)\frac{F_{Surf}(\omega_i)}{\omega_i}
\end{align*}

First term is an average of $\frac{\delta f_1}{f_1}$ weighted by a kernel $\mathcal{K}(r,r_0)=\sum_ic_i(r_0)K_{1,2}^i(r)$.

Second terms is the influence of second function on the solution of the first: weighting function $\mathcal{L}_{21}(r_0,r)=\sum_ic_i(r_0)K_{21}^i(r)$.

The third term is the influence of surface.

The coefficient $c_i(r_0)$ are selected to resmble target function, minimize contamination from $\frac{\delta f_2}{f_2}$ via $\mathcal{L}_{21}$ and minimize effect of noise:

are choosen to minimize
\begin{align*}
&\int(\sum_ic_iK_{12}^i)^2\,dr
&+\beta\int(\sum_ic_iK_{21}^1)^2\,dr\\
&+\mu\sum_{ij}c_ic_jE_{ij}
\end{align*}

$\beta$ is a parameter for contribution of second term.

\chapter{Helioseismic constrain on solar structure}

We use a SSM as starting model about which hydrostatic equations are linearized: see variational principle connecting differences between solar and the model function describing radial structure to corresponding differences in modes frequencies.

From inversion we infer the value of observables 

\begin{align*}
&Q_{\odot}=Q_{Mod}+q(\omega)&\intertext{for a given inversion procedure $\Delta\Omega$ propagate to the helioseismic value of observable $Q_{\odot}$, also we a residual dependence on starting model and regularization procedure}
\end{align*}


Asymptotic approximation for radial eigenfunction (integral equation connectin sound speed $c(r)$ to $\Omega_{nl}$) is inadequate (especially in deep interior)

\section{Helioseismological ''correction'' procedure}

\begin{itemize}
\item $\{\Omega\}$ of p-modes $\xrightarrow{\text{inversion}}Q$.
\item Solar Model $\to Q_{Mod}\to \{\Omega_{Mod}\}$.
\item $\Omega_{Mod}$ vs $\Omega_{\odot}\pm\Delta\Omega_{\odot}$: searching for correction q to solar model in order to match $\{\Omega_{Mod}+\omega(q)\}\leftrightarrow\Omega_{\odot}$. ($\omega(q)$: linear perturbation theory)
\end{itemize}

Assumptions:
\begin{align*}
&q=Q_{\odot}-Q_{Mod}\\
&\gamma=\Gamma_{\odot}-\Gamma_{Mod}&\intertext{$P, \rho$ and combination of their derivatives: connected through linearized mechanical equilibrium condition}
\end{align*}

\begin{itemize}
\item Slow variation of unknown functions
    \item Using a thermodynamical relation for $\Gamma(P,\rho,Y)$ one can eliminate one of the functions $q(x),\gamma(x), F(\Omega)$ and assuming $Y=Y_{ph}$ in convection zone and $\gamma=0$ in radiative interior
\end{itemize}


with these additional constraints the unknown function $\gamma$ is related with unknown number $Y_{ph}^{\odot}$, and chosing $U=\frac{P}{\rho}$ we write \autoref{eq:diffthermo}, where
\begin{itemize}
    \item $\delta_ru=u_{\odot}-U_{Mod}=u$.
    \item $y_{ph}=Y_{ph}^{\odot}-Y_{ph}^{mod}$.
\end{itemize}

\subsection{Outer convective zone.}

The quantities characterizing the outer part are $R_b$, $Y_{ph}$ and $c_b$, $\rho_b$ at bottom of convection zone.

\begin{itemize}
    \item He abundance. Helioseismical detemination $Y_{ph}=\numrange{0.226}{0.260}$.
    
    \item Bottom of convection zone.
    
    Transition of temperature gradient between subadiabatic and adiabatic at base of solar convection zone gives rise to clear signature in sound speed: helioseismical measurement of sound speed permits determination of base of convection zone
    
    \begin{align*}
    &\frac{R_b}{\rsun{}}=\numrange{0.710}{0.716}\\
    &c_b=\SIrange{0.221}{0.225}{\mega\meter\per\second}
    \end{align*}
    
    Lower part of convective zone is very close to being adiabatically stratified, $\Gamma\approx\frac{5}{3}$: $P\propto\rho\expy{\frac{5}{3}}$.
    
    $\rho_b$ is an indipendent quantity ($\rho(x)$ in convective zone is determined up to a scaling factor): the helioseismological determination of $\rho_b$ fixes such a factor.
    
    \begin{equation*}
        \rho_b=\SI{0.192}{\gram\per\cubic\cm}
    \end{equation*}
    
\end{itemize}

\subsection{Intermediate part ($0.2<x<0.65$)}

Isothermal sound speed $U=\frac{P}{\rho}$: recostruction of sound speed profile $c^2=\Gamma U$.

Below convective zone $\Gamma=\frac{5}{3}$ with an accuracy of \num{e-3} or better. Helioseis. determination is very accurate in this region $\frac{\Delta U}{U}\leq5 \perthousand$.

\subsection{Inner part ($x<0.2$)}



\backmatter

\printbibliography



\end{document}
